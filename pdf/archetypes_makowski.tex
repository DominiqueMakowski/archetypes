\documentclass[]{book}
\usepackage{lmodern}
\usepackage{amssymb,amsmath}
\usepackage{ifxetex,ifluatex}
\usepackage{fixltx2e} % provides \textsubscript
\ifnum 0\ifxetex 1\fi\ifluatex 1\fi=0 % if pdftex
  \usepackage[T1]{fontenc}
  \usepackage[utf8]{inputenc}
\else % if luatex or xelatex
  \ifxetex
    \usepackage{mathspec}
  \else
    \usepackage{fontspec}
  \fi
  \defaultfontfeatures{Ligatures=TeX,Scale=MatchLowercase}
\fi
% use upquote if available, for straight quotes in verbatim environments
\IfFileExists{upquote.sty}{\usepackage{upquote}}{}
% use microtype if available
\IfFileExists{microtype.sty}{%
\usepackage{microtype}
\UseMicrotypeSet[protrusion]{basicmath} % disable protrusion for tt fonts
}{}
\usepackage{hyperref}
\hypersetup{unicode=true,
            pdftitle={Uncovering the Neurocognitive Archetypes},
            pdfauthor={Dr.~Dominique Makowski},
            pdfborder={0 0 0},
            breaklinks=true}
\urlstyle{same}  % don't use monospace font for urls
\usepackage{natbib}
\bibliographystyle{apalike}
\usepackage{longtable,booktabs}
\usepackage{graphicx,grffile}
\makeatletter
\def\maxwidth{\ifdim\Gin@nat@width>\linewidth\linewidth\else\Gin@nat@width\fi}
\def\maxheight{\ifdim\Gin@nat@height>\textheight\textheight\else\Gin@nat@height\fi}
\makeatother
% Scale images if necessary, so that they will not overflow the page
% margins by default, and it is still possible to overwrite the defaults
% using explicit options in \includegraphics[width, height, ...]{}
\setkeys{Gin}{width=\maxwidth,height=\maxheight,keepaspectratio}
\IfFileExists{parskip.sty}{%
\usepackage{parskip}
}{% else
\setlength{\parindent}{0pt}
\setlength{\parskip}{6pt plus 2pt minus 1pt}
}
\setlength{\emergencystretch}{3em}  % prevent overfull lines
\providecommand{\tightlist}{%
  \setlength{\itemsep}{0pt}\setlength{\parskip}{0pt}}
\setcounter{secnumdepth}{5}
% Redefines (sub)paragraphs to behave more like sections
\ifx\paragraph\undefined\else
\let\oldparagraph\paragraph
\renewcommand{\paragraph}[1]{\oldparagraph{#1}\mbox{}}
\fi
\ifx\subparagraph\undefined\else
\let\oldsubparagraph\subparagraph
\renewcommand{\subparagraph}[1]{\oldsubparagraph{#1}\mbox{}}
\fi

%%% Use protect on footnotes to avoid problems with footnotes in titles
\let\rmarkdownfootnote\footnote%
\def\footnote{\protect\rmarkdownfootnote}

%%% Change title format to be more compact
\usepackage{titling}

% Create subtitle command for use in maketitle
\providecommand{\subtitle}[1]{
  \posttitle{
    \begin{center}\large#1\end{center}
    }
}

\setlength{\droptitle}{-2em}

  \title{Uncovering the Neurocognitive Archetypes}
    \pretitle{\vspace{\droptitle}\centering\huge}
  \posttitle{\par}
    \author{Dr.~Dominique Makowski}
    \preauthor{\centering\large\emph}
  \postauthor{\par}
      \predate{\centering\large\emph}
  \postdate{\par}
    \date{2019-11-02}

\usepackage{booktabs}
\usepackage{amsthm}
\makeatletter
\def\thm@space@setup{%
  \thm@preskip=8pt plus 2pt minus 4pt
  \thm@postskip=\thm@preskip
}
\makeatother

\begin{document}
\maketitle

{
\setcounter{tocdepth}{1}
\tableofcontents
}
\hypertarget{introduction}{%
\chapter{Introduction}\label{introduction}}

\emph{Disclaimer: This is a compilation of thoughts that might be someday used in a fictional novel. \textbf{It does not reflect any personal beliefs.}}
\textbf{Warning: This is a work in progress, i.e., currently just a collection of unstructured information.}

The idea that the mind is deeply structured along universal lines has always been fascinating to me. Although strongly criticized, or simply debunked by science (along with the rest of Jungian psychology), the idea that the mind was structurally clustered in a similar way across the time and space has endured, quitting the prison of scientific psychology to pervade other fields such as history of arts, literature and anthropology. In fact, the notion of archetypes and archetypal patterns has never been so popular, evidence being ``found'' within the redundancies of Human behaviours, myths and stories of old.

But beyond the smell of pseudoscience that emanates from this concept, do archetypes make any scientific sense? The goal of this book is to provide a new and rational perspective on archetypes, informed by biology, psychology and neuroscience.

\hypertarget{history-of-archetypes}{%
\chapter{History of Archetypes}\label{history-of-archetypes}}

\hypertarget{philosophical-roots}{%
\section{Philosophical roots}\label{philosophical-roots}}

It is common to mention Plato's Idea as the origin or philosophical basis for archetypes.

\hypertarget{tarot}{%
\section{Tarot}\label{tarot}}

The \emph{Major Arcana} consists of 22 cards:

\begin{itemize}
\tightlist
\item
  The Magician
\item
  The High Priestess
\item
  The Empress
\item
  The Emperor
\item
  The Hierophant
\item
  The Lovers
\item
  The Chariot
\item
  Strength
\item
  The Hermit
\item
  Wheel of Fortune
\item
  Justice
\item
  The Hanged Man
\item
  Death
\item
  Temperance
\item
  The Devil
\item
  The Tower
\item
  The Star
\item
  The Moon
\item
  The Sun
\item
  Judgement
\item
  The World
\item
  The Fool.
\end{itemize}

Cards from The Magician to The World are numbered in Roman numerals from I to XXI, while The Fool is the only unnumbered card, sometimes placed at the beginning of the deck as 0, or at the end as XXII.

\hypertarget{carl-jung-1875---1961}{%
\section{Carl Jung (1875 - 1961)}\label{carl-jung-1875---1961}}

Originally named ``primordial images'' (a term he borrowed from the historian of art Jacob Burckhardt), Jung understood archetypes as universal, archaic patterns or images that derive from the collective unconscious, seen as the psychic counterpart of bodily instincts \citep{feist2009theories}. He makes a distinction between \emph{archetypes-as-such}, the underlying forms from which emerge motives that he refers to as \emph{archetypal images} (e.g., the mother, the child, \ldots). These archetypal images are made \emph{manifest} (i.e., explicit and consciously accessible) as they are filled with specific content through history, culture or personal history \citep{papadopoulos2012}.

In his later life, inspired by oriental philosophies, Jung used the term \emph{unus mundus} to describe the unitary reality which underlay all manifest phenomena. Jung expanded the notion of archetypes to the physical world itself, suggesting that ``psychoid'' archetypes (the non-psychic aspect of the archetype), fundamental principles of matter and energy, are the mediators of the unus mundus.

Critically, Jung's archetypes stem out of the \emph{collective unconscious}, therefore forming a substratum common to all humanity. The collective unconscious is, in Jung's perspective, referred to as the knowledge and experiences that we share as a species. A reminiscent echo of information passed down through generations from the dawn of mankind, and even afore.

Jung also named ``archetypes'' what could be seen as parts of the psyche. In particular, he focused on five of such dimensions:

\begin{itemize}
\tightlist
\item
  The \textbf{Anima} and the \textbf{Animus} are the feminine and masculine aspects of our Psyche.
\item
  The \textbf{shadow} is the part of one's Self to which the conscious ego does not identify.
\item
  The \textbf{Persona} refers to the image that a person presents to the world. According to Jung, the Persona is ``a kind of mask, designed on the one hand to make a definite impression upon others, and on the other to conceal the true nature of the individual''.
\item
  The \textbf{Self} is the realised product of the integration of all conscious and unconscious aspects of our personality.
\end{itemize}

Interstingly, while the archetypal images (the manifest motives) are usually the main focus of interest, it is important to note that Jung himself warned against such simplification. Such ``definite mythological images of motifs {[}\ldots{]} are nothing more than conscious representations; it would be absurd to assume that such variable representations could be inherited''. Again, the true archetypes for Jung were their deeper, instinctual, fluid sources -- ``the `archaic remnants', which I call `archetypes' or `primordial images'\,'' \citep{jung1964approaching}.

Altough Jung himself did not create any ``official'' list, other authors have attributed 12 different archetypal images to Jung, organized in three overarching categories, based on a fundamental driving force. These include (\textbf{find reference}):

\begin{itemize}
\tightlist
\item
  The Ego Types

  \begin{itemize}
  \tightlist
  \item
    The Innocent
  \item
    The Orphan / Everyman
  \item
    The Hero
  \item
    The Caregiver
  \end{itemize}
\item
  The Soul Types

  \begin{itemize}
  \tightlist
  \item
    The Explorer
  \item
    The Rebel
  \item
    The Lover
  \item
    The Creator
  \end{itemize}
\item
  The Self Types

  \begin{itemize}
  \tightlist
  \item
    The Jester
  \item
    The Sage
  \item
    The Magician
  \item
    The Ruler
  \end{itemize}
\end{itemize}

Jungian archetypes cannot be fully understood without taking into account his perspective on the structure of the psyche. Shadow, Persona, \ldots{}

\hypertarget{robert-moore-1942---2016}{%
\section{Robert Moore (1942 - 2016)}\label{robert-moore-1942---2016}}

In ``King Warrior Magician Lover'', \citet{moore1991king} discuss the four primary masculine archetypes:

\begin{itemize}
\tightlist
\item
  King
\item
  Warrior
\item
  Magician
\item
  Lover
\end{itemize}

To each one of these archetypes correspond an immature version:

\begin{itemize}
\tightlist
\item
  The Divine Child
\item
  The Hero
\item
  The Precocious Child
\item
  The Oedipal Child
\end{itemize}

In the same framework, feminine (altough not directly mentioned by the original authors) are:

\begin{itemize}
\tightlist
\item
  Queen
\item
  Mother
\item
  Wise woman
\item
  Lover
\end{itemize}

\hypertarget{margaret-mark-and-carol-pearson}{%
\section{Margaret Mark and Carol Pearson}\label{margaret-mark-and-carol-pearson}}

The Hero and the Outlaw:

\begin{itemize}
\tightlist
\item
  The Innocent
\item
  The Orphan
\item
  The Hero
\item
  The Caregiver
\item
  The Explorer
\item
  The Rebel
\item
  The Lover
\item
  The Creator
\item
  The Jester
\item
  The Sage
\item
  The Magician
\item
  The Ruler
\end{itemize}

\hypertarget{recent-developpments}{%
\section{Recent Developpments}\label{recent-developpments}}

The coach Scott Jeffrey, after disguising random information as facts (``archetypes influence 99\% of human behavior''), ends up with a list of \href{https://scottjeffrey.com/archetypes-list/}{over 325} archetypes (!), suggesting that their number is even larger (``the reality is that there are thousands of archetypes. Each one possesses different behavioral patterns and subtleties''). Such fined-grained perspectives, where archetypes can be anything, are arguably not about archetypes such as conceptualised in this book (as a matter, he defined them as ``set pattern of behavior'', which is very distinct from psychological protostructures).

\hypertarget{a-scientific-archetypism}{%
\chapter{A Scientific Archetypism}\label{a-scientific-archetypism}}

Many archetype theorists have adopted a bottom-up approach, from Jung itself, exploring various cultures, myths, religions or the depths of Human's psyche (using tools of pseudo-access such as psychoanalysis or hypnosis). The archetypist's role was then to assemble and integrate the interpretations that emerged from these observations, in a theory that could be applied to explain these observations.

We will take the opposite approach. Based on scientific knowledge about evolution, biology and neuroscience, we will outline a plausible framework, that we will then test on observations.

One of the main angle of attacks could be the practical value of generic archetype. Yes, a loving mother and a fearless father are common tropes in Human existence. What is the point of creating a whole pseudo-theory around them? A collateral question would be, do they (as concepts) have any direct influence over our lives?

\hypertarget{archetypes}{%
\chapter{Archetypes}\label{archetypes}}

The main archetypes discussed in this chapter reffered to as ``axial archetypes'', as they emerge from to the axes that structure our mental system. As this system grows and complexifies, its dimensionality increases, creating new planes and providing more coordinates to navigate in this space. As a result of the densification of this matrix, archetypes become more subtler, fine-grained and ultimately fading into the unicity that is characterising our identity.

Thus, archetypes are usually paired (as they are the two spaces created separated by a line), creating a symmetric and hierarchical structure. Importantly, they can be identified, and grouped, relative to their causing axis, which is often an important distinction that the organism has to learn in order to adapt.

Thus, evidence for archetypes can be indirectly gathered by demonstrating the existence and relevance of their axis of reference.

\hypertarget{masculine-and-feminine}{%
\section{Masculine and Feminine}\label{masculine-and-feminine}}

Existing theories and models often associate archetpyes with specific genders, namely ``feminine'' and ``masculine''. While this will also be the case in the current book, the reader must be aware that we see it mainly as a flexible (and convenient) appelation. In the archetypal framework, the Masculine and Feminine adjectives do not directly refer, nor are limited, to their respective biological genders, but rather to features of the two primordial archetypes, the father and the mother (see below). That does not preclude that a masculine archetype can be embodied by a female figure and \emph{vice versa}.

Moreover, another limitation of this pseudo-sexualised perspective is that it is, to some extent, anthropomorphic or primatomorphic (or at least mammalomorphic). Indeed, in these classes and families, males often endorse the role of protection, fighting for females through which reproduction can be achieved. However, other taxa of animals, with different reproduction systems, might have these roles changed, or reversed. For instance, in ants, the ruler and the warriors are Females, whereas the males are typically exhibiting roles that would be anthropomorphically labeled as feminine. Thus, while we will keep using Masculine and Feminine in their primatal sense, these adjectives may not be universally relevant.

\hypertarget{the-two-worlds}{%
\section{The Two Worlds}\label{the-two-worlds}}

The Outside \emph{vs.} the Inside, or the external and internal worlds (or Self vs.~non-Self), is the first axis differenciating our proto-experience.

Transcendal vs.~Natural.

The newborn is a fragile and dependent entity, which safety depends on the \emph{Mother} and the \emph{Father}. These are the first

Both are vectors of \textbf{protection}. The father protects from the outside world, unknown and dangerous, while the mother protects from internal dangers by fullfilling biological needs (hunger, thirst, affection, \ldots).

These two primordial archetypes will create the primordial axis, external vs.~internal worlds.

A secondary axis could be the real vs.~the unreal.

\begin{figure}

{\centering \includegraphics[width=\textwidth]{img/protection} 

}

\caption{The Primordial Axis.}\label{fig:unnamed-chunk-2}
\end{figure}

\hypertarget{known-vs.-unknown}{%
\section{\texorpdfstring{Known \emph{vs.} Unknown}{Known vs. Unknown}}\label{known-vs.-unknown}}

Knowledge

Knower of beyond vs.~knower of within.

The child wants to know, and understand this world around and within him. From the Mother, master of nature, stems out the \emph{Feminine Knower}. From the Father, master of the transcendent, stems out the \emph{Masculine Knower}.

\hypertarget{time}{%
\section{Time}\label{time}}

\hypertarget{moral}{%
\section{Moral}\label{moral}}

For elderly figures, this is assimilated with the place in the society, whether inside (priest, matron) or outside (hermit and witch). For young figures, this reflects the societal and cultural judgment. The hero and the virgin are pure archetypes of virtue, whereas the wanderer and the whore are usually societal and cultural outcasts.

\hypertarget{order}{%
\section{Order}\label{order}}

At one point, the realisation comes that all these figures are subordinated to an overarching ordering and structuring force, the \emph{Ruler}. Altough this archetype has often been related or embodied by Masculine entities, it does not need to be so, as it highly depends on societal and biologoical factors (related to the culture or Specie concerned). For instance, due to the cultural and political system of the western world (for instance, feodalism), the ruler position was generally connected to some level of physical protection offered (hence the king was usually glorified as a great warrior).

\hypertarget{the-artist}{%
\section{The artist}\label{the-artist}}

The artist is the integration of the feminine creation with the masculine transcendental nature of the object. Located at the centre, it shares aspects and potentialities of other archetypes. For instance, the artist can take uptake the purity, such as Michel-Angelo, or being the outcast, combining features of the hermit, living on the fringes of society, but also with the wanderer (as the rogue traveller and story-teller), as well as the whore () or the witch ().

\begin{figure}

{\centering \includegraphics[width=\textwidth]{img/archetypes} 

}

\caption{The main archetypes and their hierarchical structure.}\label{fig:unnamed-chunk-3}
\end{figure}

\hypertarget{the-serpent}{%
\section{The Serpent}\label{the-serpent}}

The serpent represent the cyclical nature of existence. A mother creates a child, the child becomes the mother, and the cycle restarts.

Becoming an adult means facing the serpent. Alternative ways can be to embrace (becomeà the serpent (fulfilling the archetypes and becoming the Father or the Mother), or to battle the serpent, by trying to break the cycle.

\hypertarget{light-and-shadow}{%
\section{Light and Shadow}\label{light-and-shadow}}

The stronger the light, the darker the shadow.
Idea that everybody has a ``shadow'', a collection of morally reprehensible desires and thoughts, and that its strength is proportional to the amount of light shining from the persona. Yin and Yang. True peace of mind could be achieved by reuning them (The Grey Path).

\hypertarget{archetypes-and-the-brain}{%
\chapter{Archetypes and the Brain}\label{archetypes-and-the-brain}}

Do archetypes have a biological existence?

\hypertarget{origin}{%
\section{Origin}\label{origin}}

Are the archetypes innate? No, (altough they might have been favoured by evolution); they stem out of the redunduncies of existence.

\hypertarget{archetypes-as-neurocognitive-invariants}{%
\section{Archetypes as Neurocognitive Invariants}\label{archetypes-as-neurocognitive-invariants}}

contrary to Jung, these archetypes are not originating from some collective unconscious, but rather are created and maintained by the ontology of the individual, and patterns are crystilised out of the répétition and similarity accross individual ontologies (e.g., the presnece of a fatherly and a motherly figure, etc.)

\hypertarget{neurological-substrate}{%
\section{Neurological Substrate}\label{neurological-substrate}}

Attempts have been made to map archetypes unto brain structures \citep{samuels2003jung}. For instance, \citet{rossi1977cerebral} suggested that one could locate the archetypes in the right cerebral hemisphere, based on the idea that the left hemisphe would be primarily verbal and associational, and the right primarily visuospatial and apperceptive. In light of the current neuroscientific data, these theories appear as nonsensical and naive.

That being said, if archetypes are engrammed into the cognitive system, a biological basis has to exist. One potential answer is to reframe archetypism under the predictive coding framework. As such, archetypes could be seen as meta-priors.

\hypertarget{archetypes-and-psychology}{%
\chapter{Archetypes and Psychology}\label{archetypes-and-psychology}}

The archetypal framework developped in this book is by essence reductionist, exploring the abstract core features of cognition and essentializing them to concepts of pure abstraction. As such, archetypes do not directly relate to individual psychology, nor to anything tangible.

\hypertarget{relationship-with-personality}{%
\section{Relationship with Personality}\label{relationship-with-personality}}

\hypertarget{archetypal-stories}{%
\chapter{Archetypal stories}\label{archetypal-stories}}

Common motives in religions and myths beg the question of the existence or plausibility of archetypal stories.

\hypertarget{the-origin-of-religions}{%
\chapter{The Origin of Religions}\label{the-origin-of-religions}}

It is possible that archetypes have been deified, mother (nature and creation), father (protection and sky - the unreachable world).

Archetypes might have been integrated with a will to explain terryfing or unknown phenomena, such as storm, rain, fire, day-night cycle etc.

\bibliography{book.bib,packages.bib}


\end{document}
